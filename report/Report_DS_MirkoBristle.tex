% TEMPLATE for Usenix papers, specifically to meet requirements of
%  USENIX '05
% originally a template for producing IEEE-format articles using LaTeX.
%   written by Matthew Ward, CS Department, Worcester Polytechnic Institute.
% adapted by David Beazley for his excellent SWIG paper in Proceedings,
%   Tcl 96
% turned into a smartass generic template by De Clarke, with thanks to
%   both the above pioneers
% use at your own risk.  Complaints to /dev/null.
% make it two column with no page numbering, default is 10 point

% Munged by Fred Douglis <douglis@research.att.com> 10/97 to separate
% the .sty file from the LaTeX source template, so that people can
% more easily include the .sty file into an existing document.  Also
% changed to more closely follow the style guidelines as represented
% by the Word sample file. 

% Note that since 2010, USENIX does not require endnotes. If you want
% foot of page notes, don't include the endnotes package in the 
% usepackage command, below.

\documentclass[letterpaper,twocolumn,12pt]{article}
\usepackage{usenix,epsfig,endnotes}
\usepackage{natbib}
\usepackage{amsmath}
\usepackage{algorithmicx}
\usepackage{algpseudocode}
\begin{document}


%make title bold and 14 pt font (Latex default is non-bold, 16 pt)
    \title{\Large \bf Multivariate Time Series Forecasting with Graph Neural Networks \\
    \Medium \it A Report for the Data Science Seminar
    }

    \author{
            {\rm Mirko Bristle, 14-109-576}\\
        University of Fribourg \\
        \\
        \today
    }
    \maketitle

% Use the following at camera-ready time to suppress page numbers.
% Comment it out when you first submit the paper for review.
%\thispagestyle{empty}

%
%\subsection*{Abstract}
%Your Abstract Text Goes Here.  Just a few facts.
%Whet our appetites.


    \section{Introduction}

    Prediction of time series has a strong history with nevertheless current importance. With the rise of IoT devices, and a substancial
    increase in information gathering in a wide variaty of domains the bases for the current state of the art time series forcasting was laid.
    The applicability of time series prediction can be found in several domains such as finance, electricity markets, but also several fields in science [tbc].


    This is citing \cite{wu2020connecting}


    \section{Methodical Background}
    The current report can be placed in the methodical perspective of time series forecasting while on the other hand using
    taking advantage of the more recently developed toolbox of graph neural networks (GNN).

    \subsection{Multivariate Time series forecasting (MTSF)}
    Time series forecasting has a long history [tbc].

    \textbf{ARIMA} ...

    \textbf{GP} ...

    \textbf{RNN} ...

    \textbf{LSTM}  ...

    \subsection{Graph neural networks}


    \section{MTGNN: Connecting the Dots}
    In the paper of Wu et al. \cite{wu2020connecting} two major challenges concerning the combination of GNN and multivariate
    time series forecasting are addressed.
    The graph structure in MTSF represents the relation between the different temporal features.
    A major issue is, that without prior knowledge, the relation between these features is unknown.
    Further does the graph structure differ between different data sets, what makes it impossible to have a full end-to-end framework that adjusts to the properties of the graph.
    The second challenge addresses the issue that a previously learned or explicitly supplied graph structure might not be optimal and thus,
    should be updated during the learning process.
    A step, most previously published GNNs leave a side, while focusing only on message passing.

    \subsection{Formulating the Problem}
    A $N$ dimensional multivariate timeseries consists of a serie of steps in time $z_t \in R^N$, with $z_t[i] \in R$
    being the $i^{th}$ - variable at time step $t$. The obejctive is to predict either a Q-step-away value
    $Y = \{z_{t_{P+Q}}\}$ or a sequence of M values $Y = \{z_{t_{P+1}}, z_{t_{P+2}},\dots,z_{t_{P+M}}\}$
    from a past time series with $P$ steps $X = \{z_{t_{1}}, z_{t_{2}},\dots,z_{t_{P}}\}$.

    The P past steps  $\Chi = {S_{t_1}, S_{t_2}, \dots, S_{t_P}}$ can be supplemented by $D$ auxiliary features s.d. the $i_{th}$-step
    $S_{t_i} \in R^{N+D}$ has $z_t[i]$ as the first column, followed by the auxiliary features.

    The objective is to minimize the loss of the the mapping $f: \chi \rightarrow Y$ with $L2$-regularization.

    Let $G$ be a graph consisting of nodes $V$ and edges $E$, then formally $G=(V,E)$. The vertice $v \in V$ can be
    interpreted as the $i^{th}$ - variable $z_t[i]$. Thus, the number of nodes in $G$ can be described with $N$.
    Further, $e = (u,v)$ represents an edge from node $u$ to node $v$, what represents the relation between the nodes.

    The Node Neighborhood of a node $v \in V$ can then be described as $N(v) = \{u \in V| (u,v) \in E\}$.
    The relationship of the nodes and with that the existence of edges between every two nodes $v_i,v_j \in G$
    can be described in a adjacency Matrix $A \in R^{N\timesN}$ by
    having  $A_{ij} = c > 0$ if $(v_i,v_j) \in E$. If there is no edge $(v_i,v_j) \notin E$ we write $Aij = 0$.

    \subsection{Architectual overview}
    The authors propose three major components to address the challenges. To attack the first challenge a new graph learning layer
    is introduced. The goal is to extract a first proposal of the interconnectedness of the features. The output of the model is a sparse adjacency matrix based on the supplied training data.
    Based on the graph learning layer, the spacial dependencies among the features are learned in an graph convolution module to address the over-smoothing problem [describe -> directed graphs].
    In the temporal convolution module the temporal patterns with different frequency and length should caught with a customized 1D convolution.

    \subsection{Graph Learning Layer}
    When trying to take the relationships of different variables of a time series into account, one encounters two major issues.
    First, suppose the graph $G$ is a directed graph and thus $(u,v)\in E$ might differ from $(v,u)\in E$,
    then calculating the full adjacency matrix by a common distance metric as the the euclidean distance results in time and space complexity of $O(N^2)$.
    With such an approach, forecasting a time series with a high number of dimensions is nearly impossible.

    Having a directed graph with a potentially asymmetric adjacency matrix is not far-fetched.
    In practice, having a node being influenced by an other node but not vice versa can easily be observed by the phenomenon of traffic flow.
    More generally, any signal flowing over a net of nodes over time will lead to uni-directed relations.
    Perpendicularly, most distance metrics lead to a symmetric adjacency matrix as they are mostly bi-directional.

    To address these issues the authors propose a method to extract uni-directional relationships in combination with restricting the sampling
    to only a subset of nodes for each mini-batch to alleviate the time and space complexity.

    For two learnable nodes $E_i, E_j$ the uni-directional relation is calculated as

    \begin{alignat}{1}
        M_i &= \tanh (\alpha E_i \Theta_i) \\
        M_j &= \tanh (\alpha E_j \Theta_j)  \\
        A_{ij} &= ReLU(\tanh (\alpha(M_i M_j^T - M_j M_i^T))) \label{eq:adjacency_matrix}
    \end{alignat}
    \begin{algorithm}
        \caption{Algorithm: argtopk}\label{alg:argtopk}
        \begin{algorithmic}[1]
            \For{n = 1,2, \dots, N}
                \State $idx \gets argtopk(A[n,:], 1)$
                \State $A[n, -idx] \gets 0$
            \EndFor
        \end{algorithmic}
    \end{algorithm}
    where the saturation-rate [explain] is controlled by the hyper-parameter $\alpha$ and  $\Theta_i, \Theta_j$ represent
    model parameters. The adjacency matrix $A$ is regularized by the equation \ref{eq:adjacency_matrix}
    by setting the negative counterpart of a positive valued cell $A_{ij}$ to $A_{ji} = 0$. In the algorithm argtopk \ref{alg:argtopk}
    all weights apart from the top k are set to 0 for each node.
    This means we keep only the edges between the node and its top k neighbors and make the adjacency matrix thus sparse to improve the computation in the graph convolution module.

    This in an attempt to have a flexible way fo learning stable and interpretable features during training and can still be updatable by new training data after the online training process.
    The authors refrained to update the weights dynamically over time as this prevents the model from convergence.
    To benefit from external knowledge about the graph structure of the model, it is possible to add a static node feature matrix $Z$
    which is then used to calculate the adjacency matrix with $E_1 =  E_2 = Z $.

    \subsection{Graph Concolution Module}
    In the so called graph convolutional module the authors try to integrate information about the node neighbors by introducing two mix-hop layers.

    The issue of integrating spacial information such as the relation between the nodes has been studied [TODO read and elaborate 9,2].

    The key goal of the mix-hop layer is to integrating the information of different neighbors of different distances by compromising on time and space complexity \cite{AbuElHaija}.
    More generally the neighborhood aggregation in this step is basically a variant of Laplacian smoothing \cite{leskovec2005graphs}.
    A well know difficulty is the potentially negative impact of information given by the nodes in higher levels of the signal stream such as the root of the graph.
    This can be handled by keeping a proportion of the original hidden states while at the same time integrating the information
    of the node neighborhood.
    It is also necessary to avoid that the multi layer network reaches the random walk's limit distribution.
    If not handled, the recursive information propagation along the graph structure converges towards a single point with increasing number of graph convolution layers. \cite{klicpera2018predict}.

    The authors tackled these issues by introducing an improved version of the mix-propagation layer that functions in an
    information propagation step and a information selection step.
    The information propagation step defines the $i-th$ hop with $A$ being the adjaceny matrix as
    \begin{alignat}{1}
        H^{(i)} &= \beta H_{in} + (1-\beta) \tilde{A} H^{(i-1)} \label{eq:hop}
    \end{alignat}
    where the hidden state input is set as $H^0 = H_{in}$, $\tilde{A} = \tilde{D}^{-1}(A +I)$ with $\tilde{D}_{ii}=1 + \sum_{j}^{K}A_{ij}$.
    The hyperparameter $\beta$ defines the ratio between the the amount of new information flowing in by the next hop versus the information of the original hidden state input.

    In a second step the information is selected by the parameter matrix $W^{i}$ as following
    \begin{equation}
        \begin{alignat}{1}
            H_{out} &= \mathlarger{\sum}_{i}^{K} H^i W^i  \label{eq:hop_selection}
        \end{alignat}
    \end{equation}
    In the end one of the two mix-hop layers gets run with the adjacency matrix and the other one with the transposed adjacency matrix.
    The output of the graph convolution module is then the sum of both mix-hop layers.


    \subsection{Temporal Concolution Module}
    [x]
    \subsection{The learning Algorithm}
    [x]


    \section{Experimental evaluation}


    \section{Replication Experiments}

    \subsection{Method of replication}

    \subsection{Results}


    \section{Discussion}

    \subsection{Integreation of Evaluation and Replication}

    \subsection{Contribution to time series forecasting}

    \subsection{Critique}
    \subsection{Future research}

    {\footnotesize \bibliographystyle{acm}
    \bibliography{bibliography}}


    \theendnotes

\end{document}


%A paragraph of text goes here.  Lots of text.  Plenty of interesting
%text. \\
%
%More fascinating text. Features\endnote{Remember to use endnotes, not footnotes!} galore, plethora of promises.\\
%
%\section{This is Another Section}
%
%Some embedded literal typset code might
%look like the following :
%
%{\tt \small
%\begin{verbatim}
%#include <iostream>
%using namespace std;
%main()
%{
%cout << "Hello world \n";
%return 0;
%}
%
%\end{verbatim}
%}
%
%Now we're going to cite somebody.  Watch for the cite tag.
%Here it comes~\cite{Einstein}.
%
%Lorem ipsum dolor sit amet, consectetur adipiscing elit, sed do eiusmod tempor incididunt ut labore et dolore magna aliqua. Ut enim ad minim veniam, quis nostrud exercitation ullamco laboris nisi ut aliquip ex ea commodo consequat. Duis aute irure dolor in reprehenderit in voluptate velit esse cillum dolore eu fugiat nulla pariatur. Excepteur sint occaecat cupidatat non proident, sunt in culpa qui officia deserunt mollit anim id est laborum.
%
%Lorem ipsum dolor sit amet, consectetur adipiscing elit, sed do eiusmod tempor incididunt ut labore et dolore magna aliqua. Ut enim ad minim veniam, quis nostrud exercitation ullamco laboris nisi ut aliquip ex ea commodo consequat. Duis aute irure dolor in reprehenderit in voluptate velit esse cillum dolore eu fugiat nulla pariatur. Excepteur sint occaecat cupidatat non proident, sunt in culpa qui officia deserunt mollit anim id est laborum.
%
%Lorem ipsum dolor sit amet, consectetur adipiscing elit, sed do eiusmod tempor incididunt ut labore et dolore magna aliqua. Ut enim ad minim veniam, quis nostrud exercitation ullamco laboris nisi ut aliquip ex ea commodo consequat. Duis aute irure dolor in reprehenderit in voluptate velit esse cillum dolore eu fugiat nulla pariatur. Excepteur sint occaecat cupidatat non proident, sunt in culpa qui officia deserunt mollit anim id est laborum.
%
%Lorem ipsum dolor sit amet, consectetur adipiscing elit, sed do eiusmod tempor incididunt ut labore et dolore magna aliqua. Ut enim ad minim veniam, quis nostrud exercitation ullamco laboris nisi ut aliquip ex ea commodo consequat. Duis aute irure dolor in reprehenderit in voluptate velit esse cillum dolore eu fugiat nulla pariatur. Excepteur sint occaecat cupidatat non proident, sunt in culpa qui officia deserunt mollit anim id est laborum.







